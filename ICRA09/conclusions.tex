We developed and presented a method to systematically derive
decentralized control policies for a swarm of robots performing
assembly tasks to manufacture different products in response to
varying demand. We represent the system using a multi-level modeling
methodology. The micro-continuous model is derived from the dynamics
of the robots and the assembly task and is implemented on a
physics-based simulator.  The collective behavior of the system,
including the physical interactions between robots, is abstracted to
a macro-continuous ODE model. The model incorporates parameters that
govern the stochastic control policies running on individual robots
for performing the assembly task. By tuning the parameters of the
ODE, we can also tune the performance of the assembly system. This
optimization relies on global stability properties of a specific
class of chemical reaction networks that are modeled by the
macro-continuous model. We implement the optimization as a linear
program with constraints on target amounts of parts at equilibrium,
using two possible objective functions that are based on an estimate
of the system convergence time.  We simulate the macro-continuous
model and observe that it achieves the target final product amounts
faster with optimized rates than with non-optimal rates, and that it
can quickly respond to changes in the target equilibrium.  Finally,
we map the rates onto probabilities of assembly and disassembly in
the micro-continuous model.  We find that the resulting system can
produce the target product amounts, although discrepancies arise due
to violation of the well-mixed property, low part numbers, and
failure to capture certain physical effects in the macro-continuous
model.

Our future work will focus on tuning the macro-continuous model to
match the physical system more closely, addressing in particular the
lack of spatial homogeneity due to a small number of robots and
parts. In addition, we also want to investigate the synthesis of the
discrete assembly plans and incorporate feedback into the process.
This direction draws inspiration from bio-molecular pathways in
which intermediate subassemblies or molecules can promote or inhibit
chemical reactions. We would like to be able to optimize the
discrete assembly plan by constructing feedback loops to improve the
yield rate.
